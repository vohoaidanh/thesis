\label{glossary}

%backbone
\newglossaryentry{backbone}{
	name=\texttt{backbone},
	description={Hay còn gọi \textit{mạng xương sống} trong tiếng Việt, là phần cơ bản của mô hình chịu trách nhiệm trích xuất các đặc trưng từ dữ liệu đầu vào. Các cấu trúc phổ biến thường sử dụng làm \textit{backbone} trong các nhiệm vụ liên quan đến hình ảnh gồm ResNet, VGG, và EfficientNet},
	short={backbone},
}
%
%up-sampling 
\newglossaryentry{up-sampling}{
	name=\texttt{up-sampling},
	description={Toán tử \texttt{up-sampling} có chức năng tăng kích thước của ảnh bằng cách chèn các điểm mới giữa các điểm hiện có, được dùng phổ biến trong các mô hình tạo sinh, nhằm tăng kích thước của đầu ra so với đầu vào},
	short={up-sampling},
}

%convolution
\newglossaryentry{convolution}{
	name=\texttt{convolution},
	description={Toán tử \texttt{convolution}~\cite{convolution} là một toán tử được sử dụng trong mạng tích chập, dùng để trích xuất đặc trưng của tín hiệu},
	short={convolution},
}

%histogram
\newglossaryentry{histogram}{
	name=\texttt{histogram},
	description={Là một biểu đồ cột biểu diễn sự phân bố các giá trị màu sắc hoặc mức xám trong một hình ảnh. Biểu đồ này cung cấp một cái nhìn tổng quan về tần suất xuất hiện của các giá trị màu sắc khác nhau trong ảnh, từ đó có thể hiểu rõ hơn về đặc điểm của ảnh, chẳng hạn như độ sáng, độ tương phản, và sự cân bằng màu.
	 },
	short={histogram},
}