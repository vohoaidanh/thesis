\label{glossary}

%backbone
\newglossaryentry{backbone}{
	name=\texttt{backbone},
	description={Hay còn gọi \textit{mạng xương sống} trong tiếng Việt, là phần cơ bản của mô hình chịu trách nhiệm trích xuất các đặc trưng từ dữ liệu đầu vào. Các cấu trúc phổ biến thường sử dụng làm \textit{backbone} trong các nhiệm vụ liên quan đến hình ảnh gồm ResNet, VGG, và EfficientNet},
	short={backbone},
}
%
%up-sampling 
\newglossaryentry{up-sampling}{
	name=\texttt{up-sampling},
	description={Toán tử \texttt{up-sampling} có chức năng tăng kích thước của ảnh bằng cách chèn các điểm mới giữa các điểm hiện có, được dùng phổ biến trong các mô hình tạo sinh, nhằm tăng kích thước của đầu ra so với đầu vào},
	short={up-sampling},
}

%convolution
\newglossaryentry{convolution}{
	name=\texttt{convolution},
	description={Toán tử \texttt{convolution}~\cite{convolution} là một toán tử được sử dụng trong mạng tích chập, dùng để trích xuất đặc trưng của tín hiệu},
	short={convolution},
}

%histogram
\newglossaryentry{histogram}{
	name=\texttt{histogram},
	description={Là một biểu đồ cột biểu diễn sự phân bố các giá trị màu sắc hoặc mức xám trong một hình ảnh. Biểu đồ này cung cấp một cái nhìn tổng quan về tần suất xuất hiện của các giá trị màu sắc khác nhau trong ảnh, từ đó có thể hiểu rõ hơn về đặc điểm của ảnh, chẳng hạn như độ sáng, độ tương phản, và sự cân bằng màu.
	 },
	short={histogram},
}

%bilinear
\newglossaryentry{bilinear}{
	name=\texttt{bilinear},
	description={Hay gọi đầy đủ là Bilinear interpolation  \cite{mastylo2013bilinear}- Nội suy song tuyến tính, là phương pháp được sử dụng phổ biến để phóng to hoặc thu nhỏ ảnh. Giá trị của điểm ảnh mới được tính toán dựa trên trung bình trọng số của 4 điểm ảnh lân cận theo cả trục ngang ($x$) và dọc ($y$).},
	short={bilinear},
}


\newglossaryentry{epoch}{
	name=\texttt{epoch},
	description={Một vòng lặp hoàn chỉnh qua toàn bộ tập dữ liệu huấn luyện. Trong một epoch, mô hình được cập nhật nhiều lần, mỗi lần với một batch dữ liệu},
	short={epoch}
}

\newglossaryentry{batch}{
	name=\texttt{batch},
	description={Là một tập con của dữ liệu huấn luyện được sử dụng trong một lần lan truyền tiến và lan truyền ngược trong quá trình huấn luyện. Việc chia dữ liệu thành các batch giúp tối ưu hóa bộ nhớ và tăng tốc quá trình huấn luyện thông qua tính toán song song},
	short={batch},
}

\newglossaryentry{overfitting}{
	name=\texttt{overfitting},
	description={Là hiện tượng xảy ra khi mô hình học quá kỹ các chi tiết và nhiễu trong tập huấn luyện, dẫn đến hiệu suất kém khi áp dụng cho dữ liệu chưa từng thấy. Mô hình bị overfitting sẽ có sai số huấn luyện thấp nhưng sai số kiểm tra cao},
	short={overfitting},
}

\newglossaryentry{adam}{
	name=\texttt{Adam},
	description={Thuật toán tối ưu phổ biến trong huấn luyện mạng nơ-ron, kết hợp giữa momentum và điều chỉnh tốc độ học theo từng tham số. Tên đầy đủ là Adaptive Moment Estimation},
	short={adam},
}

