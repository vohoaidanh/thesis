\chapter{KẾT LUẬN VÀ HƯỚNG PHÁT TRIỂN}
\label{Chapter5}

Luận văn đề xuất một bộ lọc đơn giản nhưng hiệu quả, được đặc tên ADOF, nhằm bắt lấy các biến đổi mức xám giữa những điểm ảnh lân cận nhau. Bằng cách coi hình ảnh như một tín hiệu số rời rạc, áp dụng kỹ thuật sai phân cục bộ, phương pháp này loại bỏ các thành phần trung bình của tín hiệu—những thành phần mang nhiều thông tin ngữ nghĩa, nhưng lại ít hữu ích trong việc phân biệt giữa hình ảnh thật và hình ảnh tổng hợp. Bộ lọc tập trung vào việc trích lọc các dấu vết tinh vi, để cung cấp cho mô hình phân loại.

Kết quả thử nghiệm cho thấy phương pháp không chỉ giảm đáng kể độ phức tạp của mô hình mà còn cải thiện cả độ chính xác lẫn khả năng tổng quát, ngay cả trên các dữ liệu chưa từng được thấy trước đó.

Ngoài ra, chúng tôi nhận thấy tiềm năng mở rộng nghiên cứu này sang các bài toán pháp y khác, như phát hiện thao tác hình ảnh, ghép nối, mở rộng ảnh và nhận diện ảnh ngụy trang. Chúng tôi dự định tiếp tục khám phá ứng dụng của phương pháp này trong các lĩnh vực đó để nâng cao hơn nữa hiệu quả trong phân tích pháp y.