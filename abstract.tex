\chapter*{
	\begin{center} 
		\fontsize{14pt}{14pt}\selectfont 
		TÓM TẮT LUẬN VĂN 
	\end{center}
}
\addcontentsline{toc}{chapter}{TÓM TẮT LUẬN VĂN}

Sự phát triển mạnh mẽ của các mô hình tạo sinh hình ảnh như Generative Adversarial Networks (GANs) và Diffusion Models đã mở ra nhiều ứng dụng sáng tạo trong đời sống, nhưng đồng thời cũng đặt ra những nguy cơ về việc lan truyền hình ảnh giả mạo. Các hình ảnh được tạo bởi mô hình máy học ngày càng tinh vi, khiến việc phát hiện bằng mắt thường trở nên khó khăn, đặc biệt trong bối cảnh thông tin số được chia sẻ rộng rãi.

Luận văn này tập trung nghiên cứu bài toán phát hiện hình ảnh tạo sinh bằng cách kết hợp các kỹ thuật tiền xử lý ảnh với mô hình học sâu. Một bộ lọc đặc biệt được thiết kế nhằm tăng cường các đặc trưng không gian-tần số, đóng vai trò làm bước tiền xử lý trước khi đưa ảnh vào mạng phân loại. Mô hình được huấn luyện để phân biệt giữa ảnh thật và ảnh giả với mục tiêu đạt độ chính xác cao và khả năng tổng quát tốt trên các nguồn ảnh tạo sinh khác nhau.

Kết quả thực nghiệm cho thấy phương pháp đề xuất có hiệu quả vượt trội so với các phương pháp cơ bản trong việc phát hiện ảnh giả tạo sinh, đặc biệt trong bối cảnh mô hình sinh ảnh ngày càng phức tạp. Luận văn cũng phân tích các thách thức hiện tại và đề xuất hướng phát triển trong tương lai để tăng cường độ tin cậy và khả năng ứng dụng thực tế của hệ thống phát hiện ảnh giả.
