\chapter*{
	\begin{center} 
		\fontsize{14pt}{14pt}\selectfont 
		TRANG THÔNG TIN LUẬN VĂN
	\end{center}
}
\addcontentsline{toc}{chapter}{TRANG THÔNG TIN LUẬN VĂN}

\begin{flushleft}
	Tên đề tài luận văn: Phát hiện hình ảnh sinh bởi mô hình tạo sinh ảnh
	
	Ngành: Trí tuệ nhân tạo
	
	Mã số ngành: xxxxxxx
	
	Họ tên học viên cao học: Võ Hoài Danh
	
	Khóa đào tạo: 32/2022
	
	Người hướng dẫn khoa học chính: TS. Lê Trung Nghĩa
	

	Cơ sở đào tạo: Trường Đại học Khoa học Tự nhiên, ĐHQG.HCM 
\end{flushleft}

\section*{1. TÓM TẮT NỘI DUNG LUẬN VĂN}
%
Sự phát triển mạnh mẽ của các mô hình tạo sinh hình ảnh như Generative Adversarial Networks~\cite{Goodfellow2014GenerativeAN} và Diffusion Models~\cite{Ho2020DenoisingDP} đã mở ra nhiều ứng dụng sáng tạo trong đời sống, nhưng đồng thời cũng đặt ra những nguy cơ về việc lan truyền hình ảnh giả mạo. Các hình ảnh được tạo bởi mô hình máy học ngày càng tinh vi, khiến việc phát hiện bằng mắt thường trở nên khó khăn, đặc biệt trong bối cảnh thông tin số được chia sẻ rộng rãi.

Luận văn này tập trung nghiên cứu bài toán phát hiện hình ảnh tạo sinh bằng cách kết hợp các kỹ thuật tiền xử lý ảnh với mô hình học sâu. Một bộ lọc đặc biệt được thiết kế nhằm tăng cường các đặc trưng không gian-tần số, đóng vai trò làm bước tiền xử lý trước khi đưa ảnh vào mạng phân loại. Mô hình được huấn luyện để phân biệt giữa ảnh thật và ảnh giả với mục tiêu đạt độ chính xác cao và khả năng tổng quát tốt trên các nguồn ảnh tạo sinh khác nhau.

Kết quả thực nghiệm cho thấy phương pháp đề xuất có hiệu quả vượt trội so với các phương pháp cơ bản trong việc phát hiện ảnh giả tạo sinh, đặc biệt trong bối cảnh mô hình sinh ảnh ngày càng phức tạp. Luận văn cũng phân tích các thách thức hiện tại và đề xuất hướng phát triển trong tương lai để tăng cường độ tin cậy và khả năng ứng dụng thực tế của hệ thống phát hiện ảnh giả.
%
%
%
\section*{2. NHỮNG KẾT QUẢ MỚI CỦA LUẬN VĂN}
Luận văn đạt được các kết quả chính sau:

\begin{enumerate}
	%
	\item Đề xuất một bộ lọc đặc trưng mới giúp cải thiện độ chính xác và giảm chi phí tính toán so với các phương pháp tiền xử lý sử dụng \gls{fft}, đồng thời vẫn duy trì hiệu suất cao. Phương pháp này thể hiện khả năng tổng quát tốt trên nhiều tập dữ liệu được sinh ra bởi các mô hình tạo sinh khác nhau.


	%
	\item Đề xuất một kiến trúc mô hình đơn giản nhưng hiệu quả cao trong bài toán phát hiện hình ảnh tạo sinh.
	%
	
	\item Áp dụng kỹ thuật \gls{fbkd} nhằm nén mô hình hiệu quả, giúp giảm 68\% kích thước (từ 1.44 triệu tham số xuống 456,771 tham số) và tài nguyên tính toán mà vẫn duy trì hiệu năng phân loại tương đương với mô hình gốc.
	
\end{enumerate}














