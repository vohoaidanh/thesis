%VIETNAM===========================================================
\chapter*{
	\begin{center} 
		\fontsize{14pt}{14pt}\selectfont 
		TRANG THÔNG TIN LUẬN VĂN
	\end{center}
}
\addcontentsline{toc}{chapter}{TRANG THÔNG TIN LUẬN VĂN}

\begin{flushleft}
	Tên đề tài luận văn: Phát hiện hình ảnh sinh bởi mô hình tạo sinh ảnh
	
	Ngành: Trí tuệ nhân tạo
	
	Mã số ngành: 8480107
	
	Họ tên học viên cao học: Võ Hoài Danh
	
	Khóa đào tạo: 32/2022
	
	Người hướng dẫn khoa học: TS. Lê Trung Nghĩa
	

	Cơ sở đào tạo: Trường Đại học Khoa học Tự nhiên, ĐHQG.HCM 
\end{flushleft}

\section*{1. TÓM TẮT NỘI DUNG LUẬN VĂN}
%
Sự phát triển mạnh mẽ của các mô hình tạo sinh hình ảnh như Generative Adversarial Networks~\cite{Goodfellow2014GenerativeAN} và Diffusion Models~\cite{Ho2020DenoisingDP} đã mở ra nhiều ứng dụng sáng tạo trong đời sống, nhưng đồng thời cũng đặt ra những nguy cơ về việc lan truyền hình ảnh giả mạo. Các hình ảnh được tạo ra bởi mô hình học máy ngày càng trở nên tinh vi, khiến cho việc phát hiện bằng mắt thường trở nên khó khăn. Trong bối cảnh thông tin số được chia sẻ nhanh chóng qua các thiết bị thông minh, việc phát hiện hình ảnh giả mạo trực tiếp trên các thiết bị có cấu hình thấp trở nên ngày càng cấp thiết.

Luận văn tập trung vào bài toán phát hiện hình ảnh tạo sinh thông qua việc kết hợp kỹ thuật tiền xử lý ảnh với mô hình học sâu. Cụ thể, luận văn đề xuất một bộ lọc đơn giản nhưng hiệu quả, giúp nâng cao hiệu suất, độ chính xác của mô hình học sâu, bộ lọc được áp dụng ở bước tiền xử lý hình ảnh trước khi ảnh được đưa vào mạng phân loại.

Kết quả thực nghiệm cho thấy phương pháp đề xuất đạt hiệu quả vượt trội so với nhiều phương pháp hiện có trong phát hiện ảnh giả tạo sinh. Bên cạnh đó, luận văn cũng xây dựng một kiến trúc mạng đơn giản và áp dụng kỹ thuật \gls{fbkd} để nén mô hình, giúp giảm đáng kể kích thước và tài nguyên tính toán mà vẫn duy trì độ chính xác so với mô hình gốc. Hướng nghiên cứu của luận văn đặc biệt chú trọng đến việc phát triển các mô hình nhỏ gọn, phù hợp để triển khai trên các thiết bị có cấu hình thấp như điện thoại thông minh hoặc thiết bị nhúng.
%
%
%
\section*{2. NHỮNG KẾT QUẢ MỚI CỦA LUẬN VĂN}
Luận văn đạt được các kết quả chính sau:

\begin{enumerate}
	%
	\item Đề xuất một khối tiền xử lý hình ảnh mới giúp cải thiện độ chính xác và giảm chi phí tính toán so với các phương pháp tiền xử lý sử dụng \gls{fft}, đồng thời vẫn duy trì hiệu suất cao. Phương pháp này thể hiện khả năng tổng quát tốt trên nhiều tập dữ liệu được sinh ra bởi các mô hình tạo sinh khác nhau.


	%
	\item Đề xuất một kiến trúc mô hình đơn giản nhưng hiệu quả cao trong bài toán phát hiện hình ảnh tạo sinh.
	%
	
	\item Áp dụng kỹ thuật \gls{fbkd} nhằm nén mô hình hiệu quả, giúp giảm 68\% kích thước (từ 1.44 triệu tham số xuống 456,771 tham số) và tài nguyên tính toán mà vẫn duy trì hiệu năng phân loại tương đương với mô hình gốc.
	
\end{enumerate}

%\vspace{4\baselineskip}
%\begin{table}[h]
%	\begin{adjustbox}{max width =\textwidth}
%		\begin{tabular}{p{8.44cm}p{8.4cm}}
%			\multicolumn{1}{p{8.44cm}}{
%				\centering \textbf{TẬP THỂ CÁN BỘ HƯỚNG DẪN} \newline
%				\centering
%%				(Ký tên, họ tên) \newline
%			} &
%			\multicolumn{1}{p{8.4cm}}{
%				\centering \textbf{HỌC VIÊN CAO HỌC} \newline
%				\centering
%%				(Ký tên, họ tên) \newline
%			} \\
%		\end{tabular}
%	\end{adjustbox}
%\end{table}
%\vspace{2\baselineskip}
%\begin{center}
%	\textbf{XÁC NHẬN CỦA CƠ SỞ ĐÀO TẠO}
%\end{center}
%\begin{center}
%	\textbf{HIỆU TRƯỞNG}
%\end{center}

%ENGLISH=================================================================

\chapter*{
	\begin{center} 
		\fontsize{14pt}{14pt}\selectfont 
		THESIS INFORMATION
	\end{center}
}
\addcontentsline{toc}{chapter}{THESIS INFORMATION PAGE}

\begin{flushleft}
	Thesis title: Detection of Synthetic Images Generated by Generative Models
	
	Speciality: Artificial Intelligence
	
	Speciality code: 8480107
	
	Name of Master Student: Võ Hoài Danh
	
	Academic yea: 32/2022
	
	Supervisor: Dr. Lê Trung Nghĩa
	
	At: VNUHCM - University of Science
\end{flushleft}

\section*{1. SUMMARY}
%
The rapid development of image generative models such as Generative Adversarial Networks~\cite{Goodfellow2014GenerativeAN} and Diffusion Models~\cite{Ho2020DenoisingDP} has opened up many creative applications in real life, but also raised concerns about the spread of fake images. Machine-generated images have become increasingly sophisticated, making visual detection challenging. In the context of digital information rapidly shared via smart devices, detecting fake images directly on low-resource devices has become increasingly urgent.

This thesis focuses on the problem of detecting generated images by combining image preprocessing techniques with deep learning models. Specifically, the thesis proposes a simple but effective filter that improves the accuracy and performance of deep learning models, applied during the image preprocessing step before feeding images into the classification network.

Experimental results show that the proposed method achieves superior performance compared to many existing methods in detecting synthetic images. Additionally, the thesis develops a simple network architecture and applies \gls{fbkd} technique to compress the model, significantly reducing model size and computational resources while maintaining comparable accuracy to the original model. The research direction particularly emphasizes developing lightweight models suitable for deployment on low-resource devices such as smartphones or embedded systems.
%
%
%
\section*{2. NOVELTY OF THESIS}
The thesis achieves the following main results:

\begin{enumerate}
	%
	\item Proposes a novel characteristic filter that improves accuracy and reduces computational cost compared to preprocessing methods using \gls{fft}, while maintaining high performance. This method demonstrates good generalization across multiple datasets generated by various generative models.
	
	%
	\item Proposes a simple yet highly effective model architecture for the problem of detecting generated images.
	%
	
	\item Applies \gls{fbkd} technique for efficient model compression, reducing size by 68\% (from 1.44 million parameters to 456,771 parameters) and computational resources while maintaining classification performance equivalent to the original model.
	
\end{enumerate}
%
%\vspace{4\baselineskip}
%\begin{table}[h]
%	\begin{adjustbox}{max width =\textwidth}
%		\begin{tabular}{p{8.44cm}p{8.4cm}}
%			\multicolumn{1}{p{8.44cm}}{
%				\centering \textbf{SUPERVISOR} \newline
%				\centering
%%				(Signature, full name) \newline
%			} &
%			\multicolumn{1}{p{8.4cm}}{
%				\centering \textbf{Master STUDENT} \newline
%				\centering
%%				(Signature, full name) \newline
%			} \\
%		\end{tabular}
%	\end{adjustbox}
%\end{table}
%\vspace{2\baselineskip}
%\begin{center}
%	\textbf{CERTIFICATION \\ UNIVERSITY OF SCIENCE}
%\end{center}
%\begin{center}
%	\textbf{PRESIDENT}
%\end{center}









